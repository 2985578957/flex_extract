%% Generated by Sphinx.
\def\sphinxdocclass{report}
\documentclass[letterpaper,10pt,english]{sphinxmanual}
\ifdefined\pdfpxdimen
   \let\sphinxpxdimen\pdfpxdimen\else\newdimen\sphinxpxdimen
\fi \sphinxpxdimen=.75bp\relax

\PassOptionsToPackage{warn}{textcomp}
\usepackage[utf8]{inputenc}
\ifdefined\DeclareUnicodeCharacter
% support both utf8 and utf8x syntaxes
\edef\sphinxdqmaybe{\ifdefined\DeclareUnicodeCharacterAsOptional\string"\fi}
  \DeclareUnicodeCharacter{\sphinxdqmaybe00A0}{\nobreakspace}
  \DeclareUnicodeCharacter{\sphinxdqmaybe2500}{\sphinxunichar{2500}}
  \DeclareUnicodeCharacter{\sphinxdqmaybe2502}{\sphinxunichar{2502}}
  \DeclareUnicodeCharacter{\sphinxdqmaybe2514}{\sphinxunichar{2514}}
  \DeclareUnicodeCharacter{\sphinxdqmaybe251C}{\sphinxunichar{251C}}
  \DeclareUnicodeCharacter{\sphinxdqmaybe2572}{\textbackslash}
\fi
\usepackage{cmap}
\usepackage[T1]{fontenc}
\usepackage{amsmath,amssymb,amstext}
\usepackage{babel}
\usepackage{times}
\usepackage[Bjarne]{fncychap}
\usepackage{sphinx}

\fvset{fontsize=\small}
\usepackage{geometry}

% Include hyperref last.
\usepackage{hyperref}
% Fix anchor placement for figures with captions.
\usepackage{hypcap}% it must be loaded after hyperref.
% Set up styles of URL: it should be placed after hyperref.
\urlstyle{same}
\addto\captionsenglish{\renewcommand{\contentsname}{Table of Contents:}}

\addto\captionsenglish{\renewcommand{\figurename}{Fig.\@ }}
\makeatletter
\def\fnum@figure{\figurename\thefigure{}}
\makeatother
\addto\captionsenglish{\renewcommand{\tablename}{Table }}
\makeatletter
\def\fnum@table{\tablename\thetable{}}
\makeatother
\addto\captionsenglish{\renewcommand{\literalblockname}{Listing}}

\addto\captionsenglish{\renewcommand{\literalblockcontinuedname}{continued from previous page}}
\addto\captionsenglish{\renewcommand{\literalblockcontinuesname}{continues on next page}}
\addto\captionsenglish{\renewcommand{\sphinxnonalphabeticalgroupname}{Non-alphabetical}}
\addto\captionsenglish{\renewcommand{\sphinxsymbolsname}{Symbols}}
\addto\captionsenglish{\renewcommand{\sphinxnumbersname}{Numbers}}

\addto\extrasenglish{\def\pageautorefname{page}}

\setcounter{tocdepth}{2}



\title{flex\_extract Documentation}
\date{Mar 07, 2019}
\release{7.1 alpha}
\author{Anne Philipp and Leopold Haimberger}
\newcommand{\sphinxlogo}{\vbox{}}
\renewcommand{\releasename}{Release}
\makeindex
\begin{document}

\pagestyle{empty}
\sphinxmaketitle
\pagestyle{plain}
\sphinxtableofcontents
\pagestyle{normal}
\phantomsection\label{\detokenize{index::doc}}


flex\_extract is a software to retrieve meteorological fields from the European Centre for Medium-Range Weather Forecasts (ECMWF) Mars archive to server as input files for the FLEXTRA/FLEXPART Atmospheric Transport Modelling system.

All third-party software and libraries required by flex\_extract are open source and free of charge.



\begin{sphinxadmonition}{note}{Note:}
!{[}{]}{[}/documentation/Sphinx/source/\_static/cc-by-40.png{]}
This work is licensed under the Creative Commons Attribution 4.0
International License. To view a copy of this license, visit
\sphinxurl{http://creativecommons.org/licenses/by/4.0/} or send a letter to
Creative Commons, PO Box 1866, Mountain View, CA 94042, USA.
\end{sphinxadmonition}


\chapter{Downloads}
\label{\detokenize{downloads:downloads}}\label{\detokenize{downloads::doc}}
Describe download options of flex\_extract ….
\begin{quote}

UNDER CONSTRUCTION
\end{quote}


\section{Download tar-balls}
\label{\detokenize{downloads/tar_balls:download-tar-balls}}\label{\detokenize{downloads/tar_balls::doc}}\begin{quote}

UNDER CONSTRUCTION
\end{quote}


\section{Git Repository}
\label{\detokenize{downloads/git_repo:git-repository}}\label{\detokenize{downloads/git_repo::doc}}\begin{quote}

UNDER CONSTRUCTION
\end{quote}


\section{History of Changes}
\label{\detokenize{downloads/history_changes:history-of-changes}}\label{\detokenize{downloads/history_changes::doc}}\begin{quote}

UNDER CONSTRUCTION
\end{quote}


\chapter{Installation}
\label{\detokenize{installation:installation}}\label{\detokenize{installation::doc}}\begin{quote}

UNDER CONSTRUCTION
\end{quote}


\section{Requirements}
\label{\detokenize{installation/requirements:requirements}}\label{\detokenize{installation/requirements::doc}}\begin{quote}

UNDER CONSTRUCTION
\end{quote}


\section{Operating modes}
\label{\detokenize{installation/oper_modes:operating-modes}}\label{\detokenize{installation/oper_modes::doc}}\begin{quote}

UNDER CONSTRUCTION
\end{quote}


\section{Installation test}
\label{\detokenize{installation/test_install:installation-test}}\label{\detokenize{installation/test_install::doc}}\begin{quote}

UNDER CONSTRUCTION
\end{quote}


\chapter{Program Structure}
\label{\detokenize{program_structure:program-structure}}\label{\detokenize{program_structure::doc}}
Describe structure of flex\_extract ….
\begin{quote}

UNDER CONSTRUCTION
\end{quote}


\section{Program Overview}
\label{\detokenize{program_structure/prog_overview:program-overview}}\label{\detokenize{program_structure/prog_overview::doc}}\begin{quote}

UNDER CONSTRUCTION
\end{quote}


\section{Program Components}
\label{\detokenize{program_structure/prog_components:program-components}}\label{\detokenize{program_structure/prog_components::doc}}\begin{quote}

UNDER CONSTRUCTION
\end{quote}


\section{Program Flow}
\label{\detokenize{program_structure/prog_flow:program-flow}}\label{\detokenize{program_structure/prog_flow::doc}}\begin{quote}

UNDER CONSTRUCTION
\end{quote}


\chapter{User Guide}
\label{\detokenize{user_guide:user-guide}}\label{\detokenize{user_guide::doc}}\begin{quote}

UNDER CONSTRUCTION
\end{quote}


\section{How to use flex\_extract}
\label{\detokenize{user_guide/how_to:how-to-use-flex-extract}}\label{\detokenize{user_guide/how_to::doc}}\begin{quote}

UNDER CONSTRUCTION
\end{quote}


\section{CONTROL-file templates}
\label{\detokenize{user_guide/control_templates:control-file-templates}}\label{\detokenize{user_guide/control_templates::doc}}\begin{quote}

UNDER CONSTRUCTION
\end{quote}


\chapter{Auto Generated Documentation}
\label{\detokenize{api:auto-generated-documentation}}\label{\detokenize{api::doc}}
\begin{sphinxShadowBox}
\begin{itemize}
\item {} 
\phantomsection\label{\detokenize{api:id1}}{\hyperref[\detokenize{api:porgrams}]{\sphinxcrossref{Porgrams}}}
\begin{itemize}
\item {} 
\phantomsection\label{\detokenize{api:id2}}{\hyperref[\detokenize{api:install}]{\sphinxcrossref{install}}}

\item {} 
\phantomsection\label{\detokenize{api:id3}}{\hyperref[\detokenize{api:submit}]{\sphinxcrossref{submit}}}

\end{itemize}

\item {} 
\phantomsection\label{\detokenize{api:id4}}{\hyperref[\detokenize{api:classes}]{\sphinxcrossref{Classes}}}
\begin{itemize}
\item {} 
\phantomsection\label{\detokenize{api:id5}}{\hyperref[\detokenize{api:controlfile}]{\sphinxcrossref{ControlFile}}}

\item {} 
\phantomsection\label{\detokenize{api:id6}}{\hyperref[\detokenize{api:ecflexpart}]{\sphinxcrossref{EcFlexpart}}}

\item {} 
\phantomsection\label{\detokenize{api:id7}}{\hyperref[\detokenize{api:gributil}]{\sphinxcrossref{GribUtil}}}

\item {} 
\phantomsection\label{\detokenize{api:id8}}{\hyperref[\detokenize{api:module-MarsRetrieval}]{\sphinxcrossref{MarsRetrieval}}}

\item {} 
\phantomsection\label{\detokenize{api:id9}}{\hyperref[\detokenize{api:uiofiles}]{\sphinxcrossref{UioFiles}}}

\end{itemize}

\item {} 
\phantomsection\label{\detokenize{api:id10}}{\hyperref[\detokenize{api:modules}]{\sphinxcrossref{Modules}}}
\begin{itemize}
\item {} 
\phantomsection\label{\detokenize{api:id11}}{\hyperref[\detokenize{api:get-mars-data}]{\sphinxcrossref{get\_mars\_data}}}

\item {} 
\phantomsection\label{\detokenize{api:id12}}{\hyperref[\detokenize{api:prepare-flexpart}]{\sphinxcrossref{prepare\_flexpart}}}

\item {} 
\phantomsection\label{\detokenize{api:id13}}{\hyperref[\detokenize{api:tools}]{\sphinxcrossref{tools}}}

\item {} 
\phantomsection\label{\detokenize{api:id14}}{\hyperref[\detokenize{api:module-disaggregation}]{\sphinxcrossref{disaggregation}}}

\end{itemize}

\end{itemize}
\end{sphinxShadowBox}


\section{Porgrams}
\label{\detokenize{api:porgrams}}

\subsection{install}
\label{\detokenize{api:install}}

\subsection{submit}
\label{\detokenize{api:submit}}

\section{Classes}
\label{\detokenize{api:classes}}

\subsection{ControlFile}
\label{\detokenize{api:controlfile}}

\subsection{EcFlexpart}
\label{\detokenize{api:ecflexpart}}

\subsection{GribUtil}
\label{\detokenize{api:gributil}}

\subsection{MarsRetrieval}
\label{\detokenize{api:module-MarsRetrieval}}\label{\detokenize{api:marsretrieval}}\index{MarsRetrieval (module)@\spxentry{MarsRetrieval}\spxextra{module}}\index{MarsRetrieval (class in MarsRetrieval)@\spxentry{MarsRetrieval}\spxextra{class in MarsRetrieval}}

\begin{fulllineitems}
\phantomsection\label{\detokenize{api:MarsRetrieval.MarsRetrieval}}\pysiglinewithargsret{\sphinxbfcode{\sphinxupquote{class }}\sphinxcode{\sphinxupquote{MarsRetrieval.}}\sphinxbfcode{\sphinxupquote{MarsRetrieval}}}{\emph{server}, \emph{public}, \emph{marsclass='EA'}, \emph{dataset=''}, \emph{type=''}, \emph{levtype=''}, \emph{levelist=''}, \emph{repres=''}, \emph{date=''}, \emph{resol=''}, \emph{stream=''}, \emph{area=''}, \emph{time=''}, \emph{step=''}, \emph{expver='1'}, \emph{number=''}, \emph{accuracy=''}, \emph{grid=''}, \emph{gaussian=''}, \emph{target=''}, \emph{param=''}}{}
Specific syntax and content for submission of MARS retrievals.

A MARS revtrieval has a specific syntax with a selection of keywords and
their corresponding values. This class provides the necessary functions
by displaying the selected parameters and their values and the actual
retrievement of the data through a mars request or a Python web api
interface. The initialization already expects all the keyword values.

A description of MARS keywords/arguments and examples of their
values can be found here:
\sphinxurl{https://software.ecmwf.int/wiki/display/UDOC/}                   Identification+keywords\#Identificationkeywords-class
\index{server (MarsRetrieval.MarsRetrieval attribute)@\spxentry{server}\spxextra{MarsRetrieval.MarsRetrieval attribute}}

\begin{fulllineitems}
\phantomsection\label{\detokenize{api:MarsRetrieval.MarsRetrieval.server}}\pysigline{\sphinxbfcode{\sphinxupquote{server}}}
This is the connection to the ECMWF data servers.
\begin{quote}\begin{description}
\item[{Type}] \leavevmode
ECMWFService or ECMWFDataServer

\end{description}\end{quote}

\end{fulllineitems}

\index{public (MarsRetrieval.MarsRetrieval attribute)@\spxentry{public}\spxextra{MarsRetrieval.MarsRetrieval attribute}}

\begin{fulllineitems}
\phantomsection\label{\detokenize{api:MarsRetrieval.MarsRetrieval.public}}\pysigline{\sphinxbfcode{\sphinxupquote{public}}}
Decides which Web API Server version is used.
\begin{quote}\begin{description}
\item[{Type}] \leavevmode
int

\end{description}\end{quote}

\end{fulllineitems}

\index{marsclass (MarsRetrieval.MarsRetrieval attribute)@\spxentry{marsclass}\spxextra{MarsRetrieval.MarsRetrieval attribute}}

\begin{fulllineitems}
\phantomsection\label{\detokenize{api:MarsRetrieval.MarsRetrieval.marsclass}}\pysigline{\sphinxbfcode{\sphinxupquote{marsclass}}}
Characterisation of dataset.
\begin{quote}\begin{description}
\item[{Type}] \leavevmode
str, optional

\end{description}\end{quote}

\end{fulllineitems}

\index{dataset (MarsRetrieval.MarsRetrieval attribute)@\spxentry{dataset}\spxextra{MarsRetrieval.MarsRetrieval attribute}}

\begin{fulllineitems}
\phantomsection\label{\detokenize{api:MarsRetrieval.MarsRetrieval.dataset}}\pysigline{\sphinxbfcode{\sphinxupquote{dataset}}}
For public datasets there is the specific naming and parameter
dataset which has to be used to characterize the type of
data.
\begin{quote}\begin{description}
\item[{Type}] \leavevmode
str, optional

\end{description}\end{quote}

\end{fulllineitems}

\index{type (MarsRetrieval.MarsRetrieval attribute)@\spxentry{type}\spxextra{MarsRetrieval.MarsRetrieval attribute}}

\begin{fulllineitems}
\phantomsection\label{\detokenize{api:MarsRetrieval.MarsRetrieval.type}}\pysigline{\sphinxbfcode{\sphinxupquote{type}}}
Determines the type of fields to be retrieved.
\begin{quote}\begin{description}
\item[{Type}] \leavevmode
str, optional

\end{description}\end{quote}

\end{fulllineitems}

\index{levtype (MarsRetrieval.MarsRetrieval attribute)@\spxentry{levtype}\spxextra{MarsRetrieval.MarsRetrieval attribute}}

\begin{fulllineitems}
\phantomsection\label{\detokenize{api:MarsRetrieval.MarsRetrieval.levtype}}\pysigline{\sphinxbfcode{\sphinxupquote{levtype}}}
Denotes type of level.
\begin{quote}\begin{description}
\item[{Type}] \leavevmode
str, optional

\end{description}\end{quote}

\end{fulllineitems}

\index{levelist (MarsRetrieval.MarsRetrieval attribute)@\spxentry{levelist}\spxextra{MarsRetrieval.MarsRetrieval attribute}}

\begin{fulllineitems}
\phantomsection\label{\detokenize{api:MarsRetrieval.MarsRetrieval.levelist}}\pysigline{\sphinxbfcode{\sphinxupquote{levelist}}}
Specifies the required levels.
\begin{quote}\begin{description}
\item[{Type}] \leavevmode
str, optional

\end{description}\end{quote}

\end{fulllineitems}

\index{repres (MarsRetrieval.MarsRetrieval attribute)@\spxentry{repres}\spxextra{MarsRetrieval.MarsRetrieval attribute}}

\begin{fulllineitems}
\phantomsection\label{\detokenize{api:MarsRetrieval.MarsRetrieval.repres}}\pysigline{\sphinxbfcode{\sphinxupquote{repres}}}
Selects the representation of the archived data.
\begin{quote}\begin{description}
\item[{Type}] \leavevmode
str, optional

\end{description}\end{quote}

\end{fulllineitems}

\index{date (MarsRetrieval.MarsRetrieval attribute)@\spxentry{date}\spxextra{MarsRetrieval.MarsRetrieval attribute}}

\begin{fulllineitems}
\phantomsection\label{\detokenize{api:MarsRetrieval.MarsRetrieval.date}}\pysigline{\sphinxbfcode{\sphinxupquote{date}}}
Specifies the Analysis date, the Forecast base date or
Observations date.
\begin{quote}\begin{description}
\item[{Type}] \leavevmode
str, optional

\end{description}\end{quote}

\end{fulllineitems}

\index{resol (MarsRetrieval.MarsRetrieval attribute)@\spxentry{resol}\spxextra{MarsRetrieval.MarsRetrieval attribute}}

\begin{fulllineitems}
\phantomsection\label{\detokenize{api:MarsRetrieval.MarsRetrieval.resol}}\pysigline{\sphinxbfcode{\sphinxupquote{resol}}}
Specifies the desired triangular truncation of retrieved data,
before carrying out any other selected post-processing.
\begin{quote}\begin{description}
\item[{Type}] \leavevmode
str, optional

\end{description}\end{quote}

\end{fulllineitems}

\index{stream (MarsRetrieval.MarsRetrieval attribute)@\spxentry{stream}\spxextra{MarsRetrieval.MarsRetrieval attribute}}

\begin{fulllineitems}
\phantomsection\label{\detokenize{api:MarsRetrieval.MarsRetrieval.stream}}\pysigline{\sphinxbfcode{\sphinxupquote{stream}}}
Identifies the forecasting system used to generate the data.
\begin{quote}\begin{description}
\item[{Type}] \leavevmode
str, optional

\end{description}\end{quote}

\end{fulllineitems}

\index{area (MarsRetrieval.MarsRetrieval attribute)@\spxentry{area}\spxextra{MarsRetrieval.MarsRetrieval attribute}}

\begin{fulllineitems}
\phantomsection\label{\detokenize{api:MarsRetrieval.MarsRetrieval.area}}\pysigline{\sphinxbfcode{\sphinxupquote{area}}}
Specifies the desired sub-area of data to be extracted.
\begin{quote}\begin{description}
\item[{Type}] \leavevmode
str, optional

\end{description}\end{quote}

\end{fulllineitems}

\index{time (MarsRetrieval.MarsRetrieval attribute)@\spxentry{time}\spxextra{MarsRetrieval.MarsRetrieval attribute}}

\begin{fulllineitems}
\phantomsection\label{\detokenize{api:MarsRetrieval.MarsRetrieval.time}}\pysigline{\sphinxbfcode{\sphinxupquote{time}}}
Specifies the time of the data in hours and minutes.
\begin{quote}\begin{description}
\item[{Type}] \leavevmode
str, optional

\end{description}\end{quote}

\end{fulllineitems}

\index{step (MarsRetrieval.MarsRetrieval attribute)@\spxentry{step}\spxextra{MarsRetrieval.MarsRetrieval attribute}}

\begin{fulllineitems}
\phantomsection\label{\detokenize{api:MarsRetrieval.MarsRetrieval.step}}\pysigline{\sphinxbfcode{\sphinxupquote{step}}}
Specifies the forecast time step from forecast base time.
\begin{quote}\begin{description}
\item[{Type}] \leavevmode
str, optional

\end{description}\end{quote}

\end{fulllineitems}

\index{expver (MarsRetrieval.MarsRetrieval attribute)@\spxentry{expver}\spxextra{MarsRetrieval.MarsRetrieval attribute}}

\begin{fulllineitems}
\phantomsection\label{\detokenize{api:MarsRetrieval.MarsRetrieval.expver}}\pysigline{\sphinxbfcode{\sphinxupquote{expver}}}
The version of the dataset.
\begin{quote}\begin{description}
\item[{Type}] \leavevmode
str, optional

\end{description}\end{quote}

\end{fulllineitems}

\index{number (MarsRetrieval.MarsRetrieval attribute)@\spxentry{number}\spxextra{MarsRetrieval.MarsRetrieval attribute}}

\begin{fulllineitems}
\phantomsection\label{\detokenize{api:MarsRetrieval.MarsRetrieval.number}}\pysigline{\sphinxbfcode{\sphinxupquote{number}}}
Selects the member in ensemble forecast run.
\begin{quote}\begin{description}
\item[{Type}] \leavevmode
str, optional

\end{description}\end{quote}

\end{fulllineitems}

\index{accuracy (MarsRetrieval.MarsRetrieval attribute)@\spxentry{accuracy}\spxextra{MarsRetrieval.MarsRetrieval attribute}}

\begin{fulllineitems}
\phantomsection\label{\detokenize{api:MarsRetrieval.MarsRetrieval.accuracy}}\pysigline{\sphinxbfcode{\sphinxupquote{accuracy}}}
Specifies the number of bits per value to be used in the
generated GRIB coded fields.
\begin{quote}\begin{description}
\item[{Type}] \leavevmode
str, optional

\end{description}\end{quote}

\end{fulllineitems}

\index{grid (MarsRetrieval.MarsRetrieval attribute)@\spxentry{grid}\spxextra{MarsRetrieval.MarsRetrieval attribute}}

\begin{fulllineitems}
\phantomsection\label{\detokenize{api:MarsRetrieval.MarsRetrieval.grid}}\pysigline{\sphinxbfcode{\sphinxupquote{grid}}}
Specifies the output grid which can be either a Gaussian grid
or a Latitude/Longitude grid.
\begin{quote}\begin{description}
\item[{Type}] \leavevmode
str, optional

\end{description}\end{quote}

\end{fulllineitems}

\index{gaussian (MarsRetrieval.MarsRetrieval attribute)@\spxentry{gaussian}\spxextra{MarsRetrieval.MarsRetrieval attribute}}

\begin{fulllineitems}
\phantomsection\label{\detokenize{api:MarsRetrieval.MarsRetrieval.gaussian}}\pysigline{\sphinxbfcode{\sphinxupquote{gaussian}}}
This parameter is deprecated and should no longer be used.
Specifies the desired type of Gaussian grid for the output.
\begin{quote}\begin{description}
\item[{Type}] \leavevmode
str, optional

\end{description}\end{quote}

\end{fulllineitems}

\index{target (MarsRetrieval.MarsRetrieval attribute)@\spxentry{target}\spxextra{MarsRetrieval.MarsRetrieval attribute}}

\begin{fulllineitems}
\phantomsection\label{\detokenize{api:MarsRetrieval.MarsRetrieval.target}}\pysigline{\sphinxbfcode{\sphinxupquote{target}}}
Specifies a file into which data is to be written after
retrieval or manipulation.
\begin{quote}\begin{description}
\item[{Type}] \leavevmode
str, optional

\end{description}\end{quote}

\end{fulllineitems}

\index{param (MarsRetrieval.MarsRetrieval attribute)@\spxentry{param}\spxextra{MarsRetrieval.MarsRetrieval attribute}}

\begin{fulllineitems}
\phantomsection\label{\detokenize{api:MarsRetrieval.MarsRetrieval.param}}\pysigline{\sphinxbfcode{\sphinxupquote{param}}}
Specifies the meteorological parameter.
\begin{quote}\begin{description}
\item[{Type}] \leavevmode
str, optional

\end{description}\end{quote}

\end{fulllineitems}

\index{data\_retrieve() (MarsRetrieval.MarsRetrieval method)@\spxentry{data\_retrieve()}\spxextra{MarsRetrieval.MarsRetrieval method}}

\begin{fulllineitems}
\phantomsection\label{\detokenize{api:MarsRetrieval.MarsRetrieval.data_retrieve}}\pysiglinewithargsret{\sphinxbfcode{\sphinxupquote{data\_retrieve}}}{}{}
Submits a MARS retrieval. Depending on the existence of
ECMWF Web-API or CDS API it is submitted via Python or a
subprocess in the Shell. The parameter for the mars retrieval
are taken from the defined class attributes.

\end{fulllineitems}

\index{display\_info() (MarsRetrieval.MarsRetrieval method)@\spxentry{display\_info()}\spxextra{MarsRetrieval.MarsRetrieval method}}

\begin{fulllineitems}
\phantomsection\label{\detokenize{api:MarsRetrieval.MarsRetrieval.display_info}}\pysiglinewithargsret{\sphinxbfcode{\sphinxupquote{display\_info}}}{}{}
Prints all class attributes and their values to the
standard output.

\end{fulllineitems}

\index{print\_infodata\_csv() (MarsRetrieval.MarsRetrieval method)@\spxentry{print\_infodata\_csv()}\spxextra{MarsRetrieval.MarsRetrieval method}}

\begin{fulllineitems}
\phantomsection\label{\detokenize{api:MarsRetrieval.MarsRetrieval.print_infodata_csv}}\pysiglinewithargsret{\sphinxbfcode{\sphinxupquote{print\_infodata\_csv}}}{\emph{inputdir}, \emph{request\_number}}{}
Write all request parameter in alpabetical order into a “csv” file.
\begin{quote}\begin{description}
\item[{Parameters}] \leavevmode\begin{itemize}
\item {} 
\sphinxstyleliteralstrong{\sphinxupquote{inputdir}} (\sphinxstyleliteralemphasis{\sphinxupquote{str}}) \textendash{} The path where all data from the retrievals are stored.

\item {} 
\sphinxstyleliteralstrong{\sphinxupquote{request\_number}} (\sphinxstyleliteralemphasis{\sphinxupquote{int}}) \textendash{} Number of mars requests for flux and non-flux data.

\end{itemize}

\end{description}\end{quote}

\end{fulllineitems}


\end{fulllineitems}



\subsection{UioFiles}
\label{\detokenize{api:uiofiles}}

\section{Modules}
\label{\detokenize{api:modules}}

\subsection{get\_mars\_data}
\label{\detokenize{api:get-mars-data}}

\subsection{prepare\_flexpart}
\label{\detokenize{api:prepare-flexpart}}

\subsection{tools}
\label{\detokenize{api:tools}}

\subsection{disaggregation}
\label{\detokenize{api:module-disaggregation}}\label{\detokenize{api:disaggregation}}\index{disaggregation (module)@\spxentry{disaggregation}\spxextra{module}}
Disaggregation of deaccumulated flux data from an ECMWF model FG field.
\begin{description}
\item[{Initially the flux data to be concerned are:}] \leavevmode\begin{itemize}
\item {} 
large-scale precipitation

\item {} 
convective precipitation

\item {} 
surface sensible heat flux

\item {} 
surface solar radiation

\item {} 
u stress

\item {} 
v stress

\end{itemize}

\end{description}

Different versions of disaggregation is provided for rainfall
data (darain, modified linear) and the surface fluxes and
stress data (dapoly, cubic polynomial).
\index{IA3() (in module disaggregation)@\spxentry{IA3()}\spxextra{in module disaggregation}}

\begin{fulllineitems}
\phantomsection\label{\detokenize{api:disaggregation.IA3}}\pysiglinewithargsret{\sphinxcode{\sphinxupquote{disaggregation.}}\sphinxbfcode{\sphinxupquote{IA3}}}{\emph{g}}{}
Interpolation with a non-negative geometric mean based algorithm.

The original grid is reconstructed by adding two sampling points in each
data series interval. This subgrid is used to keep all information during
the interpolation within the associated interval. Additionally, an advanced
monotonicity filter is applied to improve the monotonicity properties of
the series.

\begin{sphinxadmonition}{note}{Note:}
(C) Copyright 2017-2019
Sabine Hittmeir, Anne Philipp, Petra Seibert

This work is licensed under the Creative Commons Attribution 4.0
International License. To view a copy of this license, visit
\sphinxurl{http://creativecommons.org/licenses/by/4.0/} or send a letter to
Creative Commons, PO Box 1866, Mountain View, CA 94042, USA.
\end{sphinxadmonition}
\begin{quote}\begin{description}
\item[{Parameters}] \leavevmode
\sphinxstyleliteralstrong{\sphinxupquote{g}} (\sphinxstyleliteralemphasis{\sphinxupquote{list of float}}) \textendash{} Complete data series that will be interpolated having
the dimension of the original raw series.

\item[{Returns}] \leavevmode
\sphinxstylestrong{f} \textendash{} The interpolated data series with additional subgrid points.
Its dimension is equal to the length of the input data series
times three.

\item[{Return type}] \leavevmode
list of float

\end{description}\end{quote}
\subsubsection*{References}

For more information see article:
Hittmeir, S.; Philipp, A.; Seibert, P. (2017): A conservative
interpolation scheme for extensive quantities with application to the
Lagrangian particle dispersion model FLEXPART.,
Geoscientific Model Development

\end{fulllineitems}

\index{dapoly() (in module disaggregation)@\spxentry{dapoly()}\spxextra{in module disaggregation}}

\begin{fulllineitems}
\phantomsection\label{\detokenize{api:disaggregation.dapoly}}\pysiglinewithargsret{\sphinxcode{\sphinxupquote{disaggregation.}}\sphinxbfcode{\sphinxupquote{dapoly}}}{\emph{alist}}{}
Cubic polynomial interpolation of deaccumulated fluxes.

Interpolation of deaccumulated fluxes of an ECMWF model FG field
using a cubic polynomial solution which conserves the integrals
of the fluxes within each timespan.
Disaggregation is done for 4 accumluated timespans which
generates a new, disaggregated value which is output at the
central point of the 4 accumulation timespans.
This new point is used for linear interpolation of the complete
timeseries afterwards.
\begin{quote}\begin{description}
\item[{Parameters}] \leavevmode
\sphinxstyleliteralstrong{\sphinxupquote{alist}} (\sphinxstyleliteralemphasis{\sphinxupquote{list of array of float}}) \textendash{} List of 4 timespans as 2-dimensional, horizontal fields.
E.g. {[}{[}array\_t1{]}, {[}array\_t2{]}, {[}array\_t3{]}, {[}array\_t4{]}{]}

\item[{Returns}] \leavevmode
\sphinxstylestrong{nfield} \textendash{} Interpolated flux at central point of accumulation timespan.

\item[{Return type}] \leavevmode
array of float

\end{description}\end{quote}

\begin{sphinxadmonition}{note}{Note:}\begin{description}
\item[{March 2000}] \leavevmode{[}P. JAMES{]}
Original author

\item[{June 2003}] \leavevmode{[}A. BECK{]}
Adaptations

\item[{November 2015}] \leavevmode{[}Leopold Haimberger (University of Vienna){]}
Migration from Fortran to Python

\end{description}
\end{sphinxadmonition}

\end{fulllineitems}

\index{darain() (in module disaggregation)@\spxentry{darain()}\spxextra{in module disaggregation}}

\begin{fulllineitems}
\phantomsection\label{\detokenize{api:disaggregation.darain}}\pysiglinewithargsret{\sphinxcode{\sphinxupquote{disaggregation.}}\sphinxbfcode{\sphinxupquote{darain}}}{\emph{alist}}{}
Linear interpolation of deaccumulated fluxes.

Interpolation of deaccumulated fluxes of an ECMWF model FG rainfall
field using a modified linear solution which conserves the integrals
of the fluxes within each timespan.
Disaggregation is done for 4 accumluated timespans which generates
a new, disaggregated value which is output at the central point
of the 4 accumulation timespans. This new point is used for linear
interpolation of the complete timeseries afterwards.
\begin{quote}\begin{description}
\item[{Parameters}] \leavevmode
\sphinxstyleliteralstrong{\sphinxupquote{alist}} (\sphinxstyleliteralemphasis{\sphinxupquote{list of array of float}}) \textendash{} List of 4 timespans as 2-dimensional, horizontal fields.
E.g. {[}{[}array\_t1{]}, {[}array\_t2{]}, {[}array\_t3{]}, {[}array\_t4{]}{]}

\item[{Returns}] \leavevmode
\sphinxstylestrong{nfield} \textendash{} Interpolated flux at central point of accumulation timespan.

\item[{Return type}] \leavevmode
array of float

\end{description}\end{quote}

\begin{sphinxadmonition}{note}{Note:}\begin{description}
\item[{March 2000}] \leavevmode{[}P. JAMES{]}
Original author

\item[{June 2003}] \leavevmode{[}A. BECK{]}
Adaptations

\item[{November 2015}] \leavevmode{[}Leopold Haimberger (University of Vienna){]}
Migration from Fortran to Python

\end{description}
\end{sphinxadmonition}

\end{fulllineitems}



\chapter{Support}
\label{\detokenize{support:support}}\label{\detokenize{support::doc}}\begin{quote}

UNDER CONSTRUCTION
\end{quote}


\section{Ticket System}
\label{\detokenize{support/ticket_system:ticket-system}}\label{\detokenize{support/ticket_system::doc}}\begin{quote}

UNDER CONSTRUCTION
\end{quote}


\section{Mailing Lists}
\label{\detokenize{support/mailing_list:mailing-lists}}\label{\detokenize{support/mailing_list::doc}}\begin{quote}

UNDER CONSTRUCTION
\end{quote}


\section{Known Bugs and Issues}
\label{\detokenize{support/known_bugs_issues:known-bugs-and-issues}}\label{\detokenize{support/known_bugs_issues::doc}}\begin{quote}

UNDER CONSTRUCTION
\end{quote}


\section{FAQ - Frequently asked questions}
\label{\detokenize{support/faq:faq-frequently-asked-questions}}\label{\detokenize{support/faq::doc}}\begin{quote}

UNDER CONSTRUCTION
\end{quote}


\chapter{Indices and tables}
\label{\detokenize{index:indices-and-tables}}\begin{itemize}
\item {} 
\DUrole{xref,std,std-ref}{genindex}

\item {} 
\DUrole{xref,std,std-ref}{modindex}

\item {} 
\DUrole{xref,std,std-ref}{search}

\end{itemize}


\renewcommand{\indexname}{Python Module Index}
\begin{sphinxtheindex}
\let\bigletter\sphinxstyleindexlettergroup
\bigletter{d}
\item\relax\sphinxstyleindexentry{disaggregation}\sphinxstyleindexpageref{api:\detokenize{module-disaggregation}}
\indexspace
\bigletter{m}
\item\relax\sphinxstyleindexentry{MarsRetrieval}\sphinxstyleindexpageref{api:\detokenize{module-MarsRetrieval}}
\end{sphinxtheindex}

\renewcommand{\indexname}{Index}
\printindex
\end{document}